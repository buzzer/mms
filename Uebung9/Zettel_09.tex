\section*{9. \"Ubung (Abgabe: 30.06.2010, 8.30 Uhr, schriftlich)}

\subsection*{a) Entwerfen Sie ein Verfahren, das geeignet ist, die Frequenz m\"oglichst robust zu berechnen. Dabei braucht die genaue Position im Bild, an der die Schwingung beobachtet werden kann, nicht ermittelt werden.}
Die bewegte Masse bleibt bez\"uglich der (Pixel-) Gr\"o{\ss}e auf dem Bild (bereich) der Kamera gleich. Des Weiteren bewegt sich nichts vor dem (einfarbigem) Hintergrund. \\
Fourier-Transformiert man nun alle Bilder der chronologischen Folge bleibt das Frequenzspektrum \"ahnlich zwischen den Bildern.
Nur das Phasen spektrum \"andert sich mit der Bewegung der Masse. \\
Man k\"onnte die Phasen\"ahnlichkeit 2er Bilder ausnutzen um anhand ihrer Zeitstempel die Periode (und damit die Frequenz) zu bestimmen. Das Problem ist ein geeignetes \"Ahnlichkeitsma{\ss} f\"ur das Phasenspektrum. \\
W\"ahlt man einen geeigneten Bildausschnitt aus (im ``Bewegungsraum'' der Masse) kann man die Periode auch geschickt durch Messung der \"Uberdeckung messen (ideal: T=t2-t1 oder aber T=t3-t1). \\
Eine andere Idee ist mit Hilfe eines (schnellen) Kantenfinders eine Kante der Masse \"uber die Bildfolge zu folgen und so die Periode zu bestimmen. \\
Hier sollten wir uns auf eine Methode festlegen.

\subsection*{b) Wie verh\"alt sich Ihr Verfahren bei einer zeitlich ged\"ampften Schwingung?}
Me{\ss}bereich muss hierzu adaptive sein.
Mit Hilfe der Phasenspektrum-Methode ist das kein Problem. Auch ein (zuverl\"assiger) Kantenfinder passti sich der kleiner werdenden Amplitude an.

\subsection*{c) Wie verh\"alt sich Ihr Verfahren bei uneinheitlichem, aber unbewegtem Hintergrund?}
Das Phasenspektrum sollte relative unempfindlich gegen\"uber statischen Unregelm\"a{\ss}igkeiten sein.

\subsection*{d) Wei verh\"alt sich Ihr Verfahren bei der gleichzeitigen Aufnahme von mehreren Federgewichten mit unterschiedlichen Frequenzen?}
Schlecht

\subsection*{e) K\"onne Sie Ihr Verfahren so modifizieren, dass es auch die Frequenz von Kreisbewegungen und Schwingungen senkrecht zu Bildebene  erfassen kann?}
Amplituden + Phasenspektrum??