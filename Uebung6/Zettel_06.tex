\section*{6. \"Ubung (Abgabe: 09.06.2010, 8.30 Uhr, schriftlich)}

\subsection*{1. Diskutieren Sie die komplexwertige Gabor-Transformation und ihre Fourier-Transformierte. Welchen Bezug sehen Sie zu der Klasse der Hermitischen Funktionen?}

\subsection*{2. Gegeben sei die zweidimensionale isotrope Gauss-Funktion: $Gauss_{\sigma}(x,y)$}
\subsubsection*{Berechnen Sie die Ableitung dieser Funktion nach $\sigma$, sowie die zweite Ableitung der Funktion nach x und die zweite Ableitung nach y.}
\subsubsection*{Wie stehen diese Ableitungen in Beziehung? Beachte: Die Summe der zweiten Ableitungen nach x undy wird als \emph{Laplacian of Gaussion} bezeichnet und stellt einen \"ublichen Kantenoperator dar.}
\subsubsection*{Wie kann man den \emph{Laplacian of Gaussion} in einem Gaussian Skalenraum effizient approximieren?}
