\section*{6. \"Ubung (Abgabe: 09.06.2010, 8.30 Uhr, schriftlich)}

\subsection*{1. Diskutieren Sie die komplexwertige Gabor-Transformation und ihre Fourier-Transformierte. Welchen Bezug sehen Sie zu der Klasse der Hermitischen Funktionen?}

\subsection*{2. Gegeben sei die zweidimensionale isotrope Gauss-Funktion: $Gauss_{\sigma}(x,y)$}
\subsubsection*{Berechnen Sie die Ableitung dieser Funktion nach $\sigma$, sowie die zweite Ableitung der Funktion nach x und die zweite Ableitung nach y.}
1. Ableitung nach $\sigma$:
\begin{align*}
& \frac{d}{d \sigma}~\frac{1}{2 \pi \sigma^2}~e^{-\frac{x^2 + y^2}{2 \sigma^2}}\\
=& -\frac{2}{2 \pi \sigma^3}~e^{-\frac{x^2 + y^2}{2 \sigma^2}} + \frac{1}{2 \pi \sigma^2}~e^{-\frac{x^2 + y^2}{2 \sigma^2}}(-\frac{x^2 + y^2}{2}(-\frac{2}{\sigma^3}))\\
=& e^{-\frac{x^2 + y^2}{2 \sigma^2}}(-\frac{1}{\pi \sigma^3} + \frac{1}{2 \pi \sigma^2} \frac{x^2 + y^2}{\sigma^3})\\
=& e^{-\frac{x^2 + y^2}{2 \sigma^2}} (-\frac{1}{\pi \sigma^3} + \frac{x^2 + y^2}{2 \pi \sigma^5})\\
=& e^{-\frac{x^2 + y^2}{2 \sigma^2}} \frac{1}{\pi \sigma^3} (\frac{x^2 + y^2}{2 \sigma^2} - 1)
\end{align*}
2. Doppelte Ableitung nach $x$:
\begin{align*}
& \frac{d^2}{d x^2}~\frac{1}{2 \pi \sigma^2}~e^{-\frac{x^2 + y^2}{2 \sigma^2}}\\
=& \frac{d}{d x}~\frac{1}{2 \pi \sigma^2} (-\frac{2x}{2 \sigma^2})~e^{-\frac{x^2 + y^2}{2 \sigma^2}}\\
=& \frac{d}{d x}~(-\frac{1}{2 \pi \sigma^4})~x~e^{-\frac{x^2 + y^2}{2 \sigma^2}}\\
=& -\frac{1}{2 \pi \sigma^4}(e^{-\frac{x^2 + y^2}{2 \sigma^2}} + x~e^{-\frac{x^2 + y^2}{2 \sigma^2}} (-\frac{2x}{2 \sigma^2}))\\
=& \frac{1}{2 \pi \sigma^4}~e^{-\frac{x^2 + y^2}{2 \sigma^2}} (\frac{x^2}{\sigma^2} - 1)
\end{align*}
3. Doppelte Ableitung nach $y$:
\begin{align*}
& \frac{d^2}{d y^2}~\frac{1}{2 \pi \sigma^2}~e^{-\frac{x^2 + y^2}{2 \sigma^2}}\\
=& \frac{d}{d y}~\frac{1}{2 \pi \sigma^2} (-\frac{2y}{2 \sigma^2})~e^{-\frac{x^2 + y^2}{2 \sigma^2}}\\
=& \frac{d}{d y}~(-\frac{1}{2 \pi \sigma^4})~y~e^{-\frac{x^2 + y^2}{2 \sigma^2}}\\
=& -\frac{1}{2 \pi \sigma^4}(e^{-\frac{x^2 + y^2}{2 \sigma^2}} + y~e^{-\frac{x^2 + y^2}{2 \sigma^2}} (-\frac{2y}{2 \sigma^2}))\\
=& \frac{1}{2 \pi \sigma^4}~e^{-\frac{x^2 + y^2}{2 \sigma^2}} (\frac{y^2}{\sigma^2} - 1)
\end{align*}

\subsubsection*{Wie stehen diese Ableitungen in Beziehung? Beachte: Die Summe der zweiten Ableitungen nach x undy wird als \emph{Laplacian of Gaussion} bezeichnet und stellt einen \"ublichen Kantenoperator dar.}
\subsubsection*{Wie kann man den \emph{Laplacian of Gaussion} in einem Gaussian Skalenraum effizient approximieren?}