\section*{Aufgabe 1:}
Der Begriff ``Frequenz'' beschreibt die H�ufigkeit einer Wiederholung pro Zeiteinheit.\\
Frequenz l�sst sich anhand von verschiedenen Beispielen beschreiben:
\begin{itemize}
  \item Herzfrequenz als Schl�ge, die z.B. das menschliche Herz pro Minute macht.
  \item Bildfrequenz von einem Bildschirm, die beschreibt wie oft das Bild pro Sekunde aktualisiert wird.
\end{itemize}
In der Physik hat die Frequenz eine wichtige Rolle in der Beschreibung von Wellen, wie z.B. mechanische Wellen. Desweiteren wird dem Licht in der Physik auch ein Wellencharakter zugeordnet und Licht kann wie eine Welle beschrieben werden, die unter anderem auch eine Frequenz hat. So k�nnen z.B. die verschiedenen Eigenschaften (Farben, Kontraste, \ldots), die das menschliche Auge wahrnimmt mit Hilfe der Wellentheorie erkl�rt werden. 

\section*{Aufgabe 2:}
Wenn die Abbildung aus der N�he betrachtet wird, kann das Auge ein eindeutiges Bild von Einstein wahrnehmen. Dieses Bild ist jedoch an manchen Stellen unscharf/verschwommen. Wenn das Bild aus einer gr��eren Entfernung betrachtet wird, ergibt sich mit Hilfe dieser unscharfen Stellen eine ganz andere Abbildung. Statt Einstein kann in dieser Abbildung Marilyn Monroe erkannt werden.\\
Der Zusammenhang mit der Frequenz ergibt sich aus der Feststellung, dass das Licht einen Wellencharakter besitzt und die Lichtwellen, was das Auge erfasst verschiedene Wellenl�ngen und somit auch verschiedene Frequenzen haben. Durch die Graustufen im Bild wurde ein hoher Kontrast zwischen hochfrequentem Licht (schwarz) und tieffrequentem Licht (wei�) geschaffen. Wenn das Bild aus der N�he betrachtet wird, konzentriert sich das Auge mehr auf die schwarzen Teile des Bildes und erkennt besser die Feinheiten (einzelne Haare, gesichtsz�cke, \ldots). Ab einen bestimmten Punkt aus der Ferne, l�sst die Wahrnehmungsf�higkeit vom Auge nach und es werden eher tieffrequentige Lichtstrahlen erfasst und die einzelnen Feinheiten im Bild sind schlechter zu erkennen. 

\section*{Aufgabe 3:}
Die Technik, die bei den beiden Bildern verwendet wurde ist sehr �hnlich. Wenn das L�cheln der Mona Lisa genau in Detail betrachtet wird, kann eigentlich nur ein neutraler Mundausdruck beobachtet werden. Beim anschauen der gesamten Abbildung spielen andere Gesichtsz�ge und Umgebungsparameter mit ein und der Ausdruck der Mona Lisa erscheint so, als w�rde sie l�cheln.