\section*{2 �bung (Abgabe: 22.04.2010, 8.30 Uhr, schriftlich)}

\subsection*{1. Gegeben sei eine periodische Funktion �ber die Zeit, wie lauten die Fourierkoeffizienten?}
a) $f(t) = \sin(t)~\text{f�r}~t \in (-\pi, \pi)$\\
$a_0 = 0; ~ a_k = 0; ~ b_k = 1; ~ k = 1$\\\\
b) $f(t) = \cos(t)~\text{f�r}~t \in (-\pi, \pi)$\\
$a_0 = 0; ~ a_k = 1; ~ b_k = 0; ~ k = 1$\\\\
c) $f(t) = \cos(2t)~\text{f�r}~t \in (-\pi, \pi)$\\
$a_0 = 0; ~ a_k = 1; ~ b_k = 0; ~ k = 2$\\\\
d) $f(t) = 1~\text{f�r}~t \in (-\pi, \pi)$\\
$a_0 = 2; ~ a_k = 0; ~ b_k = 0; ~ k = (beliebig)$

\subsection*{2. Gegeben seien die Fourierkoeffizienten einer Funktion �ber die Zeit. Wie lautet die Funktion?}

\subsection*{3. Welche der Funktionen entspricht den Dirichletschen Bedingungen im Intervall $(-\pi,\pi)$}
a) $f(t) = sgn(t)$\\
Diese Funktion erf�llt nicht die Dirichletschen Bedingungen, da die Regel \textit{ii} verletzt werden. Es existieren keine links- oder rechtsseitigen Grenzwerte in den Punkt $t = 0$.\\\\
b) $f(t) = 1~\text{falls}~t \in \mathds{Z}~\text{, sonst}~f(t) = 0$\\
Diese Funktion erf�llt nicht die Dirichletschen Bedingungen, da die Regel \textit{i} verletzt wird. Die Funktion stellt einzelne Punkte dar, die weder stetig noch monoton sind.\\\\
c) $f(t) = 1~\text{falls}~t \in \mathds{Q}~\text{, sonst}~f(t) = 0$\\
Diese Funktion erf�llt nicht die Dirichletschen Bedingungen, da die Regel \textit{i} verletzt wird. Die Funktion stellt einzelne Punkte dar, die weder stetig noch monoton sind.\\\\
d) $f(t) = \frac{1}{t}$\\
Diese Funktion erf�llt die Dirichletschen Bedingungen. Die Funktion l�sst sich in zwei Teilintervalle aufteilen, die stetig und monoton sind und f�r den Punkt $t_0 = 0$ gilt die Regel \textit{ii}, f�r die Funktion $f(t)$ existieren der links- und rechtsseitigen Grenzwert.\\\\
e) $f(t) = \cos(\frac{1}{t})$\\
Diese Funktion erf�llt nicht die Dirichletschen Bedingungen, da die Regel \textit{i} verletzt wird. Die Funktion kann zwar in Intervalle aufgeteilt werden, die stetig und monoton sind, allerdings ist die Anzahl dieser Intervalle unendlich.\\\\ 
f) $f(t) = t~\text{mod}~1$\\
Diese Funktion erf�llt die Dirichletschen Bedingungen. Diese Funktion liefer immer $0$ als Ergebnis. Somit erf�llt sie beide Bedingungen.